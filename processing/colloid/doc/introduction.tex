\chapter{Introduction}
The package \emph{colloid} 0.9 is used to analyze the experimental data obtained from colloidal systems.

This package requires those packages to be installed first: BLAS, LAPACK, FFTW. Make sure unistd.h is available.

The data is passed as arrays, and saved and read through \emph{gdf} format. The detail format information can be seen in io.h and io.cpp, and the core of data passing is done by classes and functions in colloid\_base.h and colloid\_base.cpp. All program should include them as the data I/O and data type.

Some statistics functions is in \emph{statistics}. A few functions are added in \emph{miscellaneous}, such as sort, fitting. FFT is in a separate place so that one can able/disable FFT easily. Image analysis part is so far based on Magick++.

All other codes can be seen as packages as the extension of the core. Different packages can be added/removed without affecting others. Those are normal modes, or elasticity.

Some functions has been documented and the documents can be found in the corresponding header file.

Linux command line interfaces are stored in functions/. Run the batch\_compile and make sure the link in compile\_and\_run is correct.
