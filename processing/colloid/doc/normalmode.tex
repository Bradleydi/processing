\chapter{Normal mode}
\section{Algorithm}
The algorithm for calculating normal modes are the displacement covariance matrix: $\langle u_iu_j\rangle = k_BT [K^{-1}]_{ij}$.

To plot the modes, Gnuplot is used.

To calculate level spacing distribution and spectral rigidity, refer to the random matrix theory:
\begin{verbatim}
@BOOK{mehta04,
	author	= "M. L. Mehta", 
	title	= "Random Matrices", 
	edition	= "3rd ed.",
	publisher = "Academic Press, New York",
	year	= "2004"
}
\end{verbatim}

The transverse and longitudinal part, dispersion relations are calculated according to their definition. From the dispersion relation, one can obtain sound velocity $v=\lim_{k\to0}\omega/k$, and the elastic modulus $E=v^2\rho$.

\section{Binary file format}
\begin{verbatim}
==================
FORMAT of .mp FILE
==================
header0 header1 mp

header0: int, 'm'*256+'p'
header1: int, numbers of points * 2
mp[header1] : float, xy positions
	mp[2*i]   : x position of point i
	mp[2*i+1] : y position of point i
 
==================
FORMAT of .ev FILE
==================
header0 header1 E G

header0: int, 'e'*256+'v'
header1: int, numbers of eigenvalues
E[header1] : double, eigenvalues
G[header1*header1]  : double, eigenvectors.
	G[i*header+j] is eigenvectors corresponding eigenvalue E[i].
\end{verbatim}


\section{Codes}
Binargy I/O 
\begin{itemize}
\item  write and read .ev file: writeev, readev
\item read only a mode from .ev file: readmode
\item read only eigenvalues from .ev file: readE
\item write and read .dcm file (dynamic covariance matrix): writedcm, readdcm, writedcm3D, readdcm3D
\item write and read .mp file (mean position): writemp, readmp
\end{itemize}

Core functions:\\
\begin{verbatim}
/*! normal mode core 
 *  @param ptid
 *  	(type)    colloid_base &
 *  	(input)   [position, ...,  time, id] data
 *  @param file
 *  	(type)    const char *
 *  	(input)   filename prefix (can contain subfix) to store eigenvalues
 *  	          and eigenvectors of covariance matrix
 *  @param Remove_Drift
 *  	(type)    bool
 *  	(input)   remove drift or not
 *  	(default) remove drift
 *
 *  (input)ptid should contain same number of particles for each frame. 
 *  .ev file with prefix (input)file is generated, to store eigenvalues and
 *  eigenvectors of covariance matrix.
 */
void normal_mode(colloid_base&, const char *, double =-1, bool =true);


/*! density of states core 
 *  @param Gn
 *  	(type)    int
 *  	(input)   number of eigenvalues
 *  @param E
 *  	(type)    double *
 *  	(input)   eigenvalues
 *  @param file
 *  	(type)    const char *
 *  	(input)   filename prefix (can contain subfix) for output
 *
 *  The output is with a subfix ".dos"
 *  This program is called in
 *  	normal_mode
 *  	DOS
 */
void DOS_w(int, double *, const char *);


/*! density of states 
 *  @param ptid
 *  	(type)    colloid_base &
 *  	(input)   [position, ...,  time, id] data
 *  @param file
 *  	(type)    const char *
 *  	(input)   filename prefix (can contain subfix) for output
 *  @param Remove_Drift
 *  	(type)    bool
 *  	(input)   remove drift or not
 *  	(default) remove drift
 *
 *  (input)ptid should contain same number of particles for each frame. 
 *  The program first try to read eigen data from .ev file if found, else
 *  it will call normal_mode to generate everything.
 */
void DOS(colloid_base &, const char *, bool);

/* logarithmic binning of DOS */
void DOSlog(const char *);
void DOS_log(const char *file);

/* DOS obtatined by kernel density estimate */
void DOS_kdensity(const char *file, double h, int nbin);

/* participation ratio and smoothing */
void participation_ratio_w(int, double *, double *, const char *);
void participation_ratio(colloid_base &, const char *, bool);
void participation_ratio_smooth(const char *file, double h);

/* level sapcing distribution and spectral rigidity */
void level_separation_w(int, double *, const char *);
void level_separation(colloid_base &, const char *, bool);
void level_separation_log(const char *);
void level_spacing(const char *file, double h, unfolding_t UNFOLDING);
void level_rigidity(const char *file, double h, unfolding_t UNFOLDING);

/* amplitude distribution of a given mode */
void mode_hist_w(int, double *, int, const char *);
void mode_hist(colloid_base &, int, const char *, bool);

/* obtain covariance matrix */
void covariance(colloid_base &, const char *, bool =true);

/* spatial displacement distribution */
void uur(colloid_base &, const char *, double =-1, bool =true);

/* decomposition of a mode to transverse and longitudinal part */
void transverse_longitudinal(int&, const char *);
void transverse_longitudinal_eps(int&, const char *);

/* plot modes and ouput eps file */
void plot_mode(int&, const char *);
void plot_mode_eps(int&, const char *);

/*! project displacement to normal modes */
void projection(const char *, double =-1, bool =true);


/*! dispersion relation of the system */
void dispersion(const char *);
void plot_dispersion(const char *, int =0);

/*! inverse of dymamic covariance matrix, 
 * the hessian matrix of the system
 */
void K(const char *);
void Kr(const char *);

/*! binary IO of K */
void writeK(int &, double *, const char *);
void readK(int &, double **, const char *);

/* plot K */
void plot_K(const char *file);

/*! potential energy of each frame 
 * should pass the same argument with normal_mode
 */
void U(const char *, double =-1, bool =true);

/* soft spots */
void softspot(const char *file, int Nmode, int Nsmall);

/* spatial correlation of amplitudes of a mode */
void mode_spcorr(int n, const char *file);
void mode_spcorr_all(const char *file, double omegaRange);
\end{verbatim}
